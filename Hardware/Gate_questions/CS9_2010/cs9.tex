\documentclass[twocolumn]{article}
\usepackage{amsmath, amssymb}
\usepackage{graphicx}
\usepackage{xcolor}
\usepackage{caption}
\usepackage{geometry}
\usepackage{fancyhdr}
\usepackage{enumitem}
\usepackage{array}
\geometry{margin=0.7in}
\pagestyle{empty}

\begin{document}
\begin{figure}[t]
    \includegraphics[width=\linewidth]{iiitb_comet.jpeg} % Optional logo
    \textbf{Name: Karappagari Alekya} \\
    \textbf{Batch: 2} \\
    \textbf{ID: cometfwc024} \\
    \textbf{Date: 9th July 2025}
\end{figure}

\begin{center}
    {\LARGE \textbf{\textcolor{blue}{GATE CS 2010 – Question Number 9}}}
\end{center}

\section*{\textcolor{blue}{Question}}
The Boolean expression for the output $f$ of the multiplexer shown below is:

\begin{figure}[h]
    \centering
    \includegraphics[width=0.95\linewidth]{cs9.png}
    \caption*{\textbf{Figure: 4x1 Multiplexer with inputs and select lines}}
\end{figure}

\section*{\textcolor{blue}{Solution}}
Given a 4x1 multiplexer:
\begin{itemize}
    \item Select lines: $S_1 = P$, $S_0 = Q$
    \item Inputs:
    \begin{align*}
        I_0 &= R \\
        I_1 &= \overline{R} \\
        I_2 &= \overline{R} \\
        I_3 &= R
    \end{align*}
\end{itemize}

Using MUX expression:
\[ f = I_0 \overline{P} \overline{Q} + I_1 \overline{P} Q + I_2 P \overline{Q} + I_3 P Q \]
Substitute inputs:
\begin{align*}
    f &= R \cdot \overline{P} \cdot \overline{Q} + \overline{R} \cdot \overline{P} \cdot Q + \overline{R} \cdot P \cdot \overline{Q} + R \cdot P \cdot Q \\
      &= \overline{P} \cdot \overline{Q} \cdot R + \overline{P} \cdot Q \cdot \overline{R} + P \cdot \overline{Q} \cdot \overline{R} + P \cdot Q \cdot R
\end{align*}

This matches the expression: \[ f = P \oplus Q \oplus R \]

\section*{\textcolor{blue}{Correct Option: (B) $P \oplus Q \oplus R$}}

\section*{\textcolor{blue}{Truth Table}}

\begin{table}[h]
\centering
\renewcommand{\arraystretch}{1.2}
\begin{tabular}{|c|c|c|c|}
\hline
P & Q & R & f \\
\hline
0 & 0 & 0 & 0 \\
0 & 0 & 1 & 1 \\
0 & 1 & 0 & 1 \\
0 & 1 & 1 & 0 \\
1 & 0 & 0 & 1 \\
1 & 0 & 1 & 0 \\
1 & 1 & 0 & 0 \\
1 & 1 & 1 & 1 \\
\hline
\end{tabular}
\caption*{\textbf{Truth Table of the MUX Output}}
\end{table}

\section*{\textcolor{blue}{Hardware Implementation: Pico2W}}

\subsection*{Components Required}
\begin{table}[h]
\centering
\renewcommand{\arraystretch}{1.2}
\begin{tabular}{|>{\raggedright}p{5cm}|c|}
\hline
\textbf{Component} & \textbf{Qty} \\
\hline
Raspberry Pi Pico2W & 1 \\
Push Buttons (P, Q, R) & 3 \\
LED (Output f) & 1 \\
Resistors (220 ohm) & 4 \\
Breadboard + Jumper Wires & - \\
\hline
\end{tabular}
\caption*{\textbf{Pico2W Components}}
\end{table}

\section*{GPIO Pin Connections (Pico2W)}

\begin{table}[h!]
\centering
\renewcommand{\arraystretch}{1.3}
\begin{tabular}{|>{\raggedright}p{4cm}|c|c|}
\hline
\textbf{Component} & \textbf{GPIO Pin} & \textbf{Function} \\
\hline
Button P & GP14 & Input \\
Button Q & GP15 & Input \\
Button R & GP16 & Input \\
LED f   & GP13 & Output \\
Resistors & 10k &Stable Low\\
LED Ressitor & 220$\Omega$ & Protects LED \\
Common Ground      & GND  & Ground \\
Power Supply       & 3.3V & Logic high supply \\
\hline
\end{tabular}
\caption*{\textbf{Table: Raspberry Pi Pico2W Pin Configuration}}
\end{table}

\section*{\textcolor{blue}{Uploading Code to Pico2W}}

\begin{enumerate}
    \item Connect the Raspberry Pi Pico2W to your computer using a USB cable while holding the \textbf{BOOTSEL} button.
    \item The board appears as a USB drive on your computer.
    \item Download and drag the MicroPython \texttt{.uf2} firmware file to the Pico's USB drive.
    \item Open the \textbf{Thonny IDE} on your computer.
    \item In Thonny, go to \texttt{Tools} $\rightarrow$ \texttt{Interpreter} and select \textbf{MicroPython (Raspberry Pi Pico)}.
    \item Write or paste your Python code (logic implementation).
    \item Click \textbf{Run} or press \texttt{F5} to upload and execute the code on Pico2W.
    \item Observe the output on the LED based on button inputs.
\end{enumerate}



\section*{\textcolor{blue}{Hardware Implementation:Arduino Uno}}

\section*{GPIO Pin Connections: Arduino Uno}

\begin{table}[h]
\centering
\renewcommand{\arraystretch}{1.2}
\begin{tabular}{|c|c|c|}
\hline
\textbf{Component} & \textbf{Arduino Pin} & \textbf{Direction} \\
\hline
Push Button P & D2 & Input \\
Push Button Q & D3 & Input \\
Push Button R & D4 & Input \\
LED (Output $f$) & D5 & Output \\
GND (Common Ground) & GND & Power \\
VCC (5V) & 5V & Power \\
\hline
\end{tabular}
\caption*{\textbf{Table: Arduino Uno Pin Configuration for MUX Logic}}
\end{table}

\section*{\textcolor{blue}{Uploading Code to Arduino Uno}}

\begin{enumerate}
    \item Connect the Arduino Uno to your computer using a USB cable.
    \item Open the \textbf{Arduino IDE} (download from \texttt{arduino.cc} if not installed).
    \item Select the correct board and port:
    \begin{itemize}
        \item Go to \texttt{Tools} $\rightarrow$ \texttt{Board} $\rightarrow$ \textbf{Arduino Uno}
        \item Then \texttt{Tools} $\rightarrow$ \texttt{Port} $\rightarrow$ Select your device port
    \end{itemize}
    \item Write or paste your logic code (e.g., for NOR gate or expression implementation).
    \item Click the \textbf{Upload} button (right arrow icon) or press \texttt{Ctrl+U}.
    \item Wait for “Done uploading” message.
    \item Test using push buttons and observe output on the LED.
\end{enumerate}


\section*{\textcolor{blue}{GitHub Repository}}
\texttt{https://github.com/Alekyakuruba/fwc/tree/main/hardware}
\begin{figure}[h]
    \centering
    \includegraphics[width=0.75\linewidth]{cs9_exp.jpg}
    \caption*{\textbf{Figure: 4x1 Multiplexer implememtation}}
\end{figure}

\section*{\textcolor{blue}{Conclusion}}
This problem involved a 4x1 MUX with binary select inputs $P$ and $Q$. Substituting MUX inputs and evaluating the expression gives:
\[f = P \oplus Q \oplus R\]
The expression is confirmed by constructing the truth table and verifying with all options. Hardware implementation is feasible using both Pico2W and Arduino Uno.

\textbf{Final Answer: (B) $P \oplus Q \oplus R$}

\end{document}
