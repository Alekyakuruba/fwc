\documentclass[twocolumn]{article}
\usepackage{amsmath}
\usepackage{xcolor}
\usepackage{graphicx}
\usepackage{caption}
\usepackage{fancyhdr}
\usepackage{geometry}
\usepackage{enumitem}
\usepackage{array}
\geometry{margin=0.7in}
\pagestyle{empty}

\begin{document}

\begin{figure}[t]
    \includegraphics[width=\linewidth]{iiitb_comet.jpeg} % Replace with actual image
    \textbf{Name: Karappagari Alekya} \\
    \textbf{Batch: 2} \\
    \textbf{ID: cometfwc024} \\
    \textbf{Date: 9th July 2025}
\end{figure}

\begin{center}
    {\LARGE \textbf{\textcolor{blue}{GATE Question Paper 2010, EE Question Number 52}}}
\end{center}

\vspace{1em}
\begin{figure}[h]
    \centering
    \includegraphics[width=\linewidth]{ee52.png}
    \caption*{\textbf{Figure: Karnaugh Map for Function F}}
\end{figure}

\section*{\textcolor{blue}{Question Analysis}}
\textbf{Given:} A K-map with 3 variables: $X$, $Y$, and $Z$. Determine the minimized Boolean expression for function $F$.

\section*{Solution:}

\begin{enumerate}[label=\textbf{Step \arabic*:}]
    \item \textbf{List the Minterms where $F=1$:} \\
    From the K-map, $F=1$ at cells:
    \[
    m(0) = X'Y'Z',\quad
    m(1) = X'Y'Z,\quad
    m(3) = X'Y Z,\quad
   % m(6) = XY Z'
    \]

    \item \textbf{Group the 1s and simplify using K-map rules:}
    \begin{itemize}
        \item Group 1: $m(0)$ and $m(1)$ $\Rightarrow$ $X'Y'$
        \item Group 2: $m(1)$ and $m(3)$ $\Rightarrow$ $X'Z$
        \item Group 3: $m(6)$ alone $\Rightarrow$ $XY Z'$
    \end{itemize}

    \item \textbf{Optimal Simplification:} \\
    Combining best terms:
    \[
    F = \overline{X}Y + YZ
    \]
    This matches Option (A).
\end{enumerate}

\section*{\textcolor{blue}{Correct Option: (A)}}
\[
F = \overline{X}Y + YZ
\]

\section*{\textcolor{blue}{Truth Table for Reference}}

\begin{table}[h]
\centering
\renewcommand{\arraystretch}{1.3}
\begin{tabular}{|c|c|c|c|}
\hline
\textbf{X} & \textbf{Y} & \textbf{Z} & \textbf{F} \\
\hline
0 & 0 & 0 & 1 \\
0 & 0 & 1 & 1 \\
0 & 1 & 0 & 0 \\
0 & 1 & 1 & 1 \\
1 & 0 & 0 & 0 \\
1 & 0 & 1 & 0 \\
1 & 1 & 0 & 1 \\
1 & 1 & 1 & 0 \\
\hline
\end{tabular}
\caption*{\textbf{Table: Truth Table for F}}
\end{table}

\section*{\textcolor{blue}{Hardware Implementation }}

\textbf{Inputs:} X, Y, Z via push buttons \\
\textbf{Output:} LED to represent logic value of F

\subsection*{\textcolor{blue}{Hardware Requirements}}

\begin{table}[h]
\centering
\renewcommand{\arraystretch}{1.3}
\begin{tabular}{|c|l|}
\hline
\textbf{S.No} & \textbf{Component} \\ \hline
1 & Pico2W or Arduino Uno \\
2 & Breadboard \\
3 & Push Buttons (3x) \\
4 & LED (1x) \\
5 & Resistors: 220$\Omega$ for LED, 10k$\Omega$ for buttons \\
6 & Jumper Wires \\
7 & USB Cable \\
\hline
\end{tabular}
\caption*{\textbf{Table: Required Components}}
\end{table}

\subsection*{\textcolor{blue}{Pico2W GPIO Connection}}

\begin{table}[h]
\centering
\renewcommand{\arraystretch}{1.3}
\begin{tabular}{|c|c|c|}
\hline
\textbf{Component} & \textbf{Pico2W Pin} & \textbf{Description} \\
\hline
Button X & GP14 & Input X \\
Button Y & GP15 & Input Y \\
Button Z & GP16 & Input Z \\
LED (Output F) & GP13 & Output Logic \\
GND & GND & Common Ground \\
3.3V & 3.3V & Pull-up Supply \\
\hline
\end{tabular}
\caption*{\textbf{Table: Pico2W GPIO Mapping}}
\end{table}

\subsection*{\textcolor{blue}{Steps to Upload Code on Pico2W}}

\begin{enumerate}
    \item Hold BOOTSEL and connect Pico2W via USB.
    \item Drag MicroPython `.uf2` file to RPI-RP2 drive.
    \item Open Thonny IDE → select "MicroPython (Raspberry Pi Pico)".
    \item Write the logic expression in Python:
    \[
    F = \overline{X}Y + YZ
    \]
    \item Upload code and test LED output with buttons.
\end{enumerate}

\subsection*{\textcolor{blue}{Arduino Uno GPIO Connection}}

\begin{table}[h]
\centering
\renewcommand{\arraystretch}{1.3}
\begin{tabular}{|c|c|c|}
\hline
\textbf{Component} & \textbf{Arduino Pin} & \textbf{Description} \\
\hline
Button X & D2 & Input X \\
Button Y & D3 & Input Y \\
Button Z & D4 & Input Z \\
LED (Output F) & D13 & Output LED \\
GND & GND & Common Ground \\
5V & VCC & Pull-up for Buttons \\
\hline
\end{tabular}
\caption*{\textbf{Table: Arduino Uno Pin Mapping}}
\end{table}

\subsection*{\textcolor{blue}{Steps to Upload Code on Arduino Uno}}

\begin{enumerate}
    \item Connect Arduino Uno via USB.
    \item Open Arduino IDE.
    \item Select:
    \begin{itemize}
        \item \textbf{Board:} Arduino Uno
        \item \textbf{Port:} COMx (whichever is shown)
    \end{itemize}
    \item Write code implementing:
    \[
    F = \overline{X}Y + YZ
    \]
    \item Click \textbf{Upload}.
    \item Test with button inputs and observe LED output.
\end{enumerate}
\section*{\textcolor{blue}{GitHub Repository}}
\texttt{https://github.com/Alekyakuruba/fwc/tree/main/hardware}
\section*{\textcolor{blue}{Conclusion}}
\begin{figure}[h]
    \centering
    \includegraphics[width=0.95\linewidth]{ee52_exp.jpg}
    \caption*{\textbf{Figure: K-map implementation using pico2w}}
\end{figure}
The simplified Boolean expression:
\[
F = \overline{X}Y + YZ
\]
was verified through both \textbf{Pico2W} and \textbf{Arduino Uno} implementations using push buttons as inputs and LED for logic output.

\end{document}
