\documentclass[twocolumn]{article}
\usepackage{amsmath, amssymb}
\usepackage{graphicx}
\usepackage{xcolor}
\usepackage{caption}
\usepackage{geometry}
\usepackage{fancyhdr}
\usepackage{enumitem}
\usepackage{array}
\geometry{margin=0.7in}
\pagestyle{empty}



\begin{document}

\begin{figure}[t]
    \includegraphics[width=\linewidth]{iiitb_comet.jpeg} % Replace with your logo if needed
    \textbf{Name: Karappagari Alekya} \\
    \textbf{Batch: 2} \\
    \textbf{ID: cometfwc024} \\
    \textbf{Date: 9th July 2025}
\end{figure}

\begin{center}
    {\LARGE \textbf{\textcolor{blue}{GATE CS 2010 – Question Number 6}}}
\end{center}

\section*{\textcolor{blue}{Question}}

\begin{figure}[h]
    \centering
    \includegraphics[width=0.95\linewidth]{cs6.png}
    \caption*{\textbf{Figure: GATE CS Q6 - Minterm Expansion}}
\end{figure}

\section*{\textcolor{blue}{Given:}}

\[
f(P, Q, R) = PQ + Q\overline{R} + P\overline{R}
\]

We need to find the minterm expansion for this function.

\section*{\textcolor{blue}{Solution}}

Let us compute output \( f \) for all \( 2^3 = 8 \) combinations of \( P, Q, R \):

\begin{table}[h]
\centering
\renewcommand{\arraystretch}{1.2}
\begin{tabular}{|c|c|c|c|}
\hline
P & Q & R & $f = PQ + Q\overline{R} + P\overline{R}$ \\
\hline
0 & 0 & 0 & 0 \\
0 & 0 & 1 & 0 \\
0 & 1 & 0 & 1 \\
0 & 1 & 1 & 1 \\
1 & 0 & 0 & 1 \\
1 & 0 & 1 & 0 \\
1 & 1 & 0 & 1 \\
1 & 1 & 1 & 1 \\
\hline
\end{tabular}
\caption*{\textbf{Truth Table for f(P, Q, R)}}
\end{table}

From the truth table, \( f = 1 \) for minterms: \( m_2, m_3, m_4, m_6, m_7 \)

\[
f(P, Q, R) = \sum m(2, 3, 4, 6, 7)
\]

\section*{\textcolor{blue}{Correct Option: (D)}}

\vspace{1em}

\section*{\textcolor{blue}{Components Required}}

\begin{table}[h]
\centering
\renewcommand{\arraystretch}{1.2}
\begin{tabular}{|>{\raggedright}p{5cm}|c|}
\hline
\textbf{Component} & \textbf{Quantity} \\
\hline
Raspberry Pi Pico2W / Arduino Uno & 1 \\
Push Buttons (Inputs for P, Q, R) & 3 \\
LED (Output indicator for f) & 1 \\
220$\Omega$ Resistor & 1 \\
10k$\Omega$ Resistors & 3 \\
Breadboard & 1 \\
Jumper Wires & As required \\
USB Cable (Micro USB / USB-B) & 1 \\
\hline
\end{tabular}
\caption*{\textbf{Table: Component List}}
\end{table}

\section*{\textcolor{blue}{GPIO Connection Table}}

\subsection*{Raspberry Pi Pico2W}

\begin{table}[h]
\centering
\begin{tabular}{|c|c|c|}
\hline
Component & GPIO Pin & Role \\
\hline
Button P & GP14 & Input \\
Button Q & GP15 & Input \\
Button R & GP16 & Input \\
LED f    & GP13 & Output \\
GND      & GND  & Ground \\
3.3V     & 3.3V & Pull-up \\
\hline
\end{tabular}
\caption*{\textbf{Pico2W Connections}}
\end{table}

\subsection*{Arduino Uno}

\begin{table}[h]
\centering
\begin{tabular}{|c|c|c|}
\hline
Component & Pin & Role \\
\hline
Button P & D2 & Input \\
Button Q & D3 & Input \\
Button R & D4 & Input \\
LED f    & D5 & Output \\
GND      & GND & Ground \\
VCC      & 5V & Pull-up \\
\hline
\end{tabular}
\caption*{\textbf{Arduino Uno Connections}}
\end{table}

\section*{\textcolor{blue}{Theory: Uploading and Executing on Microcontrollers}}

Microcontrollers such as Raspberry Pi Pico2W and Arduino Uno allow us to implement digital logic circuits using software and hardware integration. The logical expressions can be evaluated in real-time by reading digital input signals from push buttons and generating output through LEDs.

\subsection*{\textcolor{blue}{Raspberry Pi Pico2W (MicroPython-based)}}

\begin{itemize}
    \item The Raspberry Pi Pico2W can be programmed using MicroPython via the Thonny IDE.
    \item To upload code:
    \begin{enumerate}
        \item Connect Pico2W to PC while holding the \texttt{BOOTSEL} button.
        \item It mounts as a USB drive named \texttt{RPI-RP2}.
        \item Drag and drop the \texttt{MicroPython UF2} firmware file.
        \item Open Thonny IDE, set the interpreter to “MicroPython (Raspberry Pi Pico)”.
        \item Write the digital logic code using \texttt{machine.Pin()} and logic operations.
        \item Click Run to upload and execute the code.
    \end{enumerate}
    \item The program continuously monitors GPIO pins connected to push buttons (representing inputs), evaluates the logic expression, and sets the output LED accordingly.
\end{itemize}

\subsection*{\textcolor{blue}{Arduino Uno (C/C++ based)}}

\begin{itemize}
    \item Arduino Uno uses the Arduino IDE and a simplified C++ syntax.
    \item To upload code:
    \begin{enumerate}
        \item Connect the Uno board using USB.
        \item Open Arduino IDE, select board as “Arduino Uno” and correct COM port.
        \item Write a sketch using \texttt{digitalRead()} for inputs and \texttt{digitalWrite()} for output.
        \item Click Upload (\texttt{→}) button to compile and flash the code.
    \end{enumerate}
    \item The Uno reads button states, evaluates the Boolean expression, and sets the LED pin high or low.
\end{itemize}

\textbf{Note:} Ensure pull-down resistors are used with push buttons to maintain a definite input logic level when the button is not pressed.

This method effectively demonstrates Boolean logic practically, helping students to visualize logic gate behavior and truth table evaluation in real-time.



\section*{\textcolor{blue}{GitHub Repository}}

All code, circuit diagrams and images are available at: \\
\texttt{https://github.com/Alekyakuruba/fwc/tree/main/hardware}
\begin{figure}[h]
    \centering
    \includegraphics[width=0.95\linewidth]{cs6_exp.jpg}
    \caption*{\textbf{Figure:Minterm Expansion Implementation}}
\end{figure}

\section*{\textcolor{blue}{Conclusion}}

This question required evaluating the minterm expansion of a composite Boolean function. We analyzed the function, derived the truth table, identified minterms where output is 1, and verified using hardware with Pico2W and Arduino Uno.

\textbf{Final Answer: (D) $m_2 + m_3 + m_4 + m_6 + m_7$}

\end{document}
