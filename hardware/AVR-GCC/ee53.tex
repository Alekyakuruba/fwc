\documentclass[twocolumn]{article}
\usepackage{amsmath}
\usepackage{xcolor}
\usepackage{graphicx}
\usepackage{float}
\usepackage{caption}
\usepackage{fancyhdr}
\usepackage{geometry}
\usepackage{enumitem}
\usepackage{array}
\geometry{margin=0.7in}
\pagestyle{empty}

\begin{document}

\begin{figure}[H]
    \begin{minipage}{0.45\textwidth}
        \includegraphics[width=\textwidth]{iiitb_comet.jpeg} % Replace with actual image path
    \end{minipage} \hfill
    \begin{minipage}{0.45\textwidth}
        \textbf{Name: Karappagari Alekya} \\
        \textbf{Batch: 2} \\
        \textbf{ID: cometfwc024} \\
        \textbf{Date: 9th July 2025}
    \end{minipage}
\end{figure}

\begin{center}
    {\LARGE \textbf{\textcolor{blue}{GATE Question Paper 2010, EC Question Number 53}}}
\end{center}

\vspace{1em}
\begin{figure}[H]
    \centering
    \includegraphics[width=0.9\linewidth]{ee53.png} % Replace with actual path
\end{figure}

\section*{\textcolor{blue}{Question Analysis}}
\textbf{We need to find the circuit that implements the minimized Boolean function:}
\[
F = \overline{X}Y + YZ
\]

\section*{\textcolor{blue}{Correct Option: (A)}}
\textbf{Explanation:} Option (A) correctly represents the logic:
\[
\overline{X}Y + YZ
\]
using AND, OR, and NOT gates.

\section*{\textcolor{blue}{Solution in Two Steps}}
\begin{enumerate}[label=\textbf{Step \arabic*:}]
    \item \textbf{Expression:} \( F = \overline{X}Y + YZ \) is in SOP form.
    \item \textbf{Option (A)} correctly implements \( \overline{X}Y \) and \( YZ \) using AND gates and combines them using an OR gate.
\end{enumerate}

\section*{\textcolor{blue}{Truth Table}}
\begin{table}[H]
\centering
\renewcommand{\arraystretch}{1.3}
\begin{tabular}{|c|c|c|c|}
\hline
X & Y & Z & F = \( \overline{X}Y + YZ \) \\
\hline
0 & 0 & 0 & 0 \\
0 & 0 & 1 & 0 \\
0 & 1 & 0 & 1 \\
0 & 1 & 1 & 1 \\
1 & 0 & 0 & 0 \\
1 & 0 & 1 & 0 \\
1 & 1 & 0 & 0 \\
1 & 1 & 1 & 1 \\
\hline
\end{tabular}
\caption*{\textbf{Truth Table of } \( F = \overline{X}Y + YZ \)}
\end{table}

\section*{\textcolor{blue}{Hardware Components}}
\begin{table}[H]
\centering
\renewcommand{\arraystretch}{1.3}
\begin{tabular}{|c|l|}
\hline
\textbf{S.No} & \textbf{Component} \\ \hline
1 & Raspberry Pi Pico2W \\
2 & Arduino Uno \\
3 & Breadboard \\
4 & Push Buttons (3x) for X, Y, Z \\
5 & LED for Output F \\
6 & Resistors (220\(\Omega\) for LED, 10k\(\Omega\) for pull-downs) \\
7 & Jumper Wires \\
8 & Micro USB Cable \\
\hline
\end{tabular}
\caption*{\textbf{Table: Required Components}}
\end{table}

\section*{\textcolor{blue}{GPIO Pin Connections (Pico2W)}}
\begin{table}[H]
\centering
\renewcommand{\arraystretch}{1.3}
\begin{tabular}{|c|c|c|}
\hline
\textbf{Component} & \textbf{Pico2W Pin} & \textbf{Description} \\
\hline
Button X & GP14 & Input X \\
Button Y & GP15 & Input Y \\
Button Z & GP16 & Input Z \\
LED (Output F) & GP13 & Output Logic \\
GND & GND & Common Ground \\
3.3V & 3.3V & Pull-up Supply to Buttons \\
\hline
\end{tabular}
\caption*{\textbf{Pico2W Pin Mapping}}
\end{table}

\section*{\textcolor{blue}{GPIO Pin Connections (Arduino Uno)}}
\begin{table}[H]
\centering
\renewcommand{\arraystretch}{1.3}
\begin{tabular}{|c|c|c|}
\hline
\textbf{Component} & \textbf{Arduino Pin} & \textbf{Description} \\
\hline
Button X & D2 & Input X \\
Button Y & D3 & Input Y \\
Button Z & D4 & Input Z \\
LED & D13 & Output Logic \\
GND & GND &  Ground \\
Vcc & 5V & supply\\
\hline
\end{tabular}
\caption*{\textbf{Arduino Pin Mapping}}
\end{table}

\section*{\textcolor{blue}{Uploading Code to Pico2W }}

\begin{enumerate}
    \item Connect the Raspberry Pi Pico2W to your computer using a USB cable while holding the \textbf{BOOTSEL} button.
    \item The board appears as a USB drive on your computer.
    \item Download and drag the MicroPython \texttt{.uf2} firmware file to the Pico's USB drive.
    \item Open the \textbf{Thonny IDE} on your computer.
    \item In Thonny, go to \texttt{Tools} $\rightarrow$ \texttt{Interpreter} and select \textbf{MicroPython (Raspberry Pi Pico)}.
    \item Write or paste your Python code (logic implementation).
    \item Click \textbf{Run} or press \texttt{F5} to upload and execute the code on Pico2W.
    \item Observe the output on the LED based on button inputs.
\end{enumerate}

\section*{\textcolor{blue}{Uploading Code to Arduino Uno }}

\begin{enumerate}
    \item Connect the Arduino Uno to your computer using a USB cable.
    \item Open the \textbf{Arduino IDE} (download from \texttt{arduino.cc} if not installed).
    \item Select the correct board and port:
    \begin{itemize}
        \item Go to \texttt{Tools} $\rightarrow$ \texttt{Board} $\rightarrow$ \textbf{Arduino Uno}
        \item Then \texttt{Tools} $\rightarrow$ \texttt{Port} $\rightarrow$ Select your device port
    \end{itemize}
    \item Write or paste your logic code (e.g., for NOR gate or expression implementation).
    \item Click the \textbf{Upload} button (right arrow icon) or press \texttt{Ctrl+U}.
    \item Wait for “Done uploading” message.
    \item Test using push buttons and observe output on the LED.
\end{enumerate}
\vspace{1em}
\section*{\textcolor{blue}{GitHub Repository}}
\texttt{https://github.com/Alekyakuruba/fwc/tree/main/hardware}
\vspace{1em}
\section*{\textcolor{blue}{Conclusion}}
Option \textbf{(A)} implements the given minimized logic expression \( F = \overline{X}Y + YZ \) correctly. Truth table and hardware testing validate the circuit on both Raspberry Pi Pico2W and Arduino Uno.
\vspace{2em}\begin{figure}[H]
    \centering
    \includegraphics[width=0.95\linewidth]{ee53_exp.jpg}
    \caption*{\textbf{Figure: Logic Circuit Implememtation}}
\end{figure}
\end{document}
